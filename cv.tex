\documentclass[10pt]{article}
\usepackage[margin=3cm]{geometry}
\usepackage[utf8]{inputenc}
\usepackage[english,slovak]{babel}
\usepackage{array, xcolor}
\definecolor{lightgray}{gray}{0.8}
\newcolumntype{L}{>{\raggedleft}p{0.14\textwidth}}
\newcolumntype{R}{p{0.8\textwidth}}
\newcommand\VRule{\color{lightgray}\vrule width 0.5pt}
\usepackage{bibentry}

\title{\bfseries\Huge Michal Chovanec}
\author{michal.nand@gmail.com, michal.chovanec@yandex.com}
\date{}
\begin{document}
\maketitle

 

\begin{minipage}[ht]{0.48\textwidth}
Ing. Michal Chovanec \\
Staničná 11\\
Lietavská Lúčka 01311\\
Slovensko
\end{minipage}



\section*{Pracovné skúsenosti}
\begin{tabular}{L!{\VRule}R}
2015--dnes& Ceit group, Žilina \\
		  & pozícia ZIMS, výskum a vývoj v oblasti robotiky,  \\ [5pt] \\

2012--2013& Scheidt\&Bachmann, Žilina \\
		  & pozícia Bahn, zabepečenie železničných priecestí (programovanie, Visual C++),  \\
		  & VŠ prax, 6 mesiacov \\

\end{tabular}


\section*{Vzdelanie}
\begin{tabular}{L!{\VRule}R}
2013--dnes & Fakulta riadenia a informatiky, Aplikovaná informatika, PhD\\[5pt] \\

2011--2013& Fakulta riadenia a informatiky, Počítačové inžinierstvo, Ing. \\
	& Téma diplomovej práce : Operačný systém pre mikrokontroléry s jadrom cortex-m3\\[5pt] \\
2007--2011& Fakulta riadenia a informatiky, Počítačové inžinierstvo, Bc.\\
\end{tabular}

\section*{Počítačové znalosti}
\begin{tabular}{L!{\VRule}R}
C & expert \\
C++, Ruby & pokročilý \\
VHLD, Java & základy \\
nástroje & GNU GCC, G++, Gnuplot, Sublime, OpenMPI, Linux, Latex \\
mikrokontroléry & ARM Cortex M0..M4, MSP430, AVR \\
technológie & NRF70, Xbee, Cuda, Xilinx FPGA \\
ďalšie & systémy reálneho času, umelá inteligencia, robotika, inerciálna navigácia, teória riadenia, adaptívne a učiace sa systémy
\end{tabular}


\section*{Ocenenia}
\begin{tabular}{L!{\VRule}R}
2013 & Ing. štúdium ukončené s vyznamenaním \\
 & Cena dekana za najlepšiu diplomovú prácu \\
 & ACM Certificate Gallery of the best \\
 & Cena Soitu za prácu súvisiacu s otvoreným softvérom, 3. miesto \\
2011 & Cena dekana za najlepšiu bakalársku prácu
\end{tabular}

\bibliographystyle{plain}
\nobibliography{publication.bib}
\section*{Publikácie}
\begin{tabular}{L!{\VRule}R}

2012 & $[1]$ Preemptívny multitasking pre mikrokontroléry s jadrom ARM Cortex M3 Michal Chovanec. - 2012 In: Otvorený softvér vo vzdelávaní, výskume a v IT riešeniach S. 27-32 zborník príspevkov medzinárodnej konferencie OSSConf 2012 2.-4. júla 2013 Žilina, Slovensko Bratislava Spoločnosť pre otvorené informačné technológie 2012, ISBN 978-80-970457-2-2 \\[5pt]
2013 & $[2]$ Akcelerometrické meranie výstrelu z luku Michal Chovanec a Jaroslav Múčka. - 2013 In: Otvorený softvér vo vzdelávaní, výskume a v IT riešeniach S. 39-46 zborník príspevkov medzinárodnej konferencie OSSConf 2013 2.-4. júla 2013 Žilina, Slovensko Bratislava Spoločnosť pre otvorené informačné technológie 2013, ISBN 978-80-970457-3-9 \\[5pt]
& $[3]$ Wireless sensor networks for intelligent transportation systems Michal Hodoň, Juraj Miček, Michal Chovanec. - 2013 In: IEEE CommSoft E-Letters Vol. 2, no. 1, (2013), online, s. 3-8 elektronický zdroj \\[5pt]
& $[4]$ Intelligent traffic-safety mirror, M. Hodon, M. Chovanec, M. Hyben, 2013 In: Studia Informatica Universalis - 2013, volume 11/1 \\[5pt]
2014 & $[5]$ Universal synchronization algorithm for wireless sensor networks - "FUSA algorithm" / Michal Chovanec ... [et al.].
In: FedCSIS : proceedings of the 2014 federated conference on Computer science and information systems : September 7-10, 2014, Warsaw, Poland. - Los Alamitos; Warsaw: IEEE; Polskie Towarzystwo Informatyczne, 2014. - ISBN 978-83-60810-61-3. - S. 1001-1007.
 \\[5pt]
& $[6]$ Tiny low-power WSN node for the vehicle detection [Jednoduchý energeticky-efektívny nód bezdôtovej senzorovej siete určený na detekciu automobilov] / Michal Chovanec, Michal Hodon and Lukas Cechovic. \\[5pt]
& $[7]$ Investigation of the gyro-sensor contribution to the straight movement of vehicle [Analýza vplyvu gyroskopického senzora pri priamom pohybe vozidla] / Michal Hodoň, Michal Chovanec.  \\[5pt]
2015 & $[8]$ Required value classification using Kohonen neural network = Klasifikácia žiadanej hodnoty Kohonenovou neurónovou sieťou / Michal Chovanec.
In: Otvorený softvér vo vzdelávaní, výskume a v IT riešeniach : zborník príspevkov medzinárodnej konferencie OSSConf 2015 : 1.-3. júla 2015 Žilina, Slovensko. - Žilina: Žilinská univerzita, 2015. - ISBN 978-80-970457-7-7. \\[5pt]


\end{tabular}

\section*{Záujmy}
robotika (umelá inteligancia, riadenie v reálnom čase),
outdoor (turistika, beh, bouldering, jaskyniarstvo, prežitie v divočine a riešenie život ohrozujúcich situacií),
bojové umenia (lukostrelba, aikidó, kenjutsu), hudba, joga


\end{document}
